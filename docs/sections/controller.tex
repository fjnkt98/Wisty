\section{速度制御のための入力信号}

前章では4輪メカナムホイールロボットの逆運動学及び順運動学を導いた.
ロボットを操縦する際は,速度ベクトル$\vec{x} = (\dot{x}, \dot{y}, \dot{\theta})^T$を入力として,逆運動学により各ホイールの速度を計算するのが一般的だが,入力できる速度には制限がある.
メカナムホイールのような全方位移動ロボットのコントローラとしては3軸のジョイスティックが用いられるのが一般的であるが,それぞれのジョイスティックをそのまま各速度に対応させるのは適切ではない.
そこで,コントローラからの3軸入力を,ロボットへの入力速度ベクトルへとマッピングする処理が必要となる.

\subsection{入力速度の制約式}

ロボットの移動速度はホイールの最大回転速度により制限される.式\ref{eq:inv3}を見ると分かるように,ロボットへの入力速度$\dot{x}$,$\dot{y}$,$\dot{\theta}$は,その絶対値の和がホイールの出せる最大回転速度を超えるような組み合わせになってはならない.
ここで,各ホイールの性能は全て同等とし,ホイールが出せる最大角速度を$\omega_{\text{max}}$,ホイールの最大周速度を$V_{\text{max}} = r \omega_{\text{max}}$とすると,式\ref{eq:constraint}に示す入力速度の制約式が得られる.

\begin{equation}
  V_{\text{max}} = |\dot{x}| + |\dot{y}| + 2l|\dot{\theta}|
  \label{eq:constraint}
\end{equation}

ここで,$\hat{\dot{\theta}} = 2l\dot{\theta}$として式\ref{eq:constraint}に代入すると,式\ref{eq:constraint}は図に示すような$\dot{x}\dot{y}\hat{\dot{\theta}}$座標系における正八面体の内部領域を表す.
ロボットへの入力速度ベクトル$\vec{x} = (\dot{x}, \dot{y}, \dot{\theta})^T$をこの座標系の位置ベクトルであるとしたとき,位置ベクトルは正八面体の内部に存在していなければならない.

\begin{figure}[h]
  \centering
  \includegraphics[width=120truemm, clip]{images/constraint.pdf}
  \caption{Constraint Area of Input Velocity Vector}
  \label{fig:constraint}
\end{figure}

\subsection{入力信号から速度ベクトルへのマッピング}